\documentclass{article}
\usepackage[utf8]{inputenc}
\usepackage[english]{babel}

\usepackage{minted}

\title{%
  Object oriented Programming (ECM1410) CA3 \\
  \large Printout}
\date{March 2021}

\begin{document}

\maketitle

\section{Classes}

\subsection{Account}
\begin{minted}
[
frame=lines,
framesep=2mm,
baselinestretch=1.2,
fontsize=\footnotesize,
linenos,
obeytabs=true,
tabsize=2
]
{java}
package socialmedia;

import java.io.Serializable;
import java.util.HashSet;
import java.util.Set;

/**
 * This class contains all the methods and attributes of the Account and therefore
 * is responsible for handling all the details related to a user on the Social Media
 * Platform. 
 * <p>
 * An Account has a unique ID (sequencially incremennting acording to num 
 * of users on the system), a string handle (username) and a description similar to 
 * a bio. 
 * <p>
 * The class itself has a static value for its SerialVersionUID used when 
 * serializing the platform, a num of accounts which tracks the number of accounts
 * created and a HashSet of all usernames (handles).
 * 
 * @author 700008432
 * @author 690033172
 * @version 1.0
 */
public class Account implements Serializable {

    /**
     * The SerialVersionUID which represents the class version.
     */
    private static final long serialVersionUID = 8225229775036968396L;

    /**
     * This is the unique sequential identifier of the account.
     */
    private int id;

    /**
     * This is the handle attributed to the account.
     */
    private String handle;

    /**
     * This is the description that the user can add to their account.
     */
    private String description;

    /**
     * This is the number of posts, including endorsements and replies the
     * account has.
     */
    private int numPosts = 0;

    /**
     * This is the number of endorsements the account has recieved.
     */
    private int numEndorsements = 0;

    /**
     * This is the static set of usernames in the system, it contains
     * all the handles of each user within the system.
     */
    private static final Set<String> usernames = new HashSet<String>();

    /**
     * This is a static integer value which records the number of accounts
     * in the system starting at 0.
     */
    private static int numAccounts = 0;

    /**
     * This is the constructor method for an Account, it takes in the parameter handle which
     * will be the username of the account. This constructor is for when the user does not
     * want to add a description on the creation of their account.
     *  
     * @param handle    The username the user wants to use.
     * 
     * @throws IllegalHandleException   Thrown if the handle is already in use.
     * @throws InvalidHandleException   Thrown if the handle is over 30 characters, 
     *                                  is empty, or contains white space.
     */
    public Account(String handle) 
                        throws IllegalHandleException,
                        InvalidHandleException{
        if (!usernames.add(handle)) {
            throw new IllegalHandleException("Handle is already in use in the system.");
        }
        if (handle.length() > 30) {
            throw new InvalidHandleException("Your handle is longer than 30 characters.");
        }
        if (handle.contains(" ")) {
            throw new InvalidHandleException("The handle cannot contain white space.");
        }
        if (handle.equals("")) {
            throw new InvalidHandleException("The handle cannot be empty.");
        }
        this.handle = handle;
        this.id = numAccounts++;
    }

    /**
     * This is the constructor method for an Account, it takes in the parameters handle which
     * will be the username of the account, and the description which will be connected to the
     * account.
     *  
     * @param handle        The username the user wants to use.
     * @param description   The description the user wants to display.
     * 
     * @throws IllegalHandleException   Thrown if the handle is already in use.
     * @throws InvalidHandleException   Thrown if the handle is over 30 characters, 
     *                                  is empty, or contains white space.
     */
    public Account(String handle, String description) 
                            throws IllegalHandleException,
                            InvalidHandleException {
        if (!usernames.add(handle)) {
            throw new IllegalHandleException("Handle is already in use in the system. "); 
        }
        if (handle.length() > 30) {
            throw new InvalidHandleException("Your handle is longer than 30 characters.");
        }
        if (handle.contains(" ")) {
            throw new InvalidHandleException("The handle cannot contain white space.");
        }
        if (handle.equals("")) {
            throw new InvalidHandleException("The handle cannot be empty.");
        }
        this.handle = handle;
        this.description = description;
        this.id = numAccounts++;
    }

    /**
     * This method resets the number of posts to 0.
     */
    public static void resetNumAccounts() {
        numAccounts = 0;
    }

    /**
     * This method is for allowing the user the change their handle after creating their
     * account. It does all the checks to be sure there are no other users with the new handle.
     * 
     * @param newHandle     This is the handle that the user wants to change to.
     * 
     * @throws InvalidHandleException   This exception is thrown if the handle is under 30 characters,
     *                                  is blank, or contains white space.
     */
    public void changeHandle(String newHandle) 
                            throws InvalidHandleException {
        if (newHandle.length() > 30) {
			throw new InvalidHandleException("Your handle is longer than 30 characters.");
		}
		if (newHandle.contains(" ")) {
			throw new InvalidHandleException("The handle cannot contain white space.");
		}
		if (newHandle.equals("")) {
			throw new InvalidHandleException("The handle cannot be empty.");
		}
        
        this.handle = newHandle;
    }
    
    /**
     * This method allows the user to update their account description.
     * 
     * @param description The new description that they want to add.
     */
    public void updateDescription(String description) {
        this.description = description;
    }

    /**
     * This method adds one to the number of posts, including endorsements and replies the 
     * account has.
     */
    public void addNumPost() {
        numPosts++;
    }

    /**
     * This method adds one to the number of endorsements the account has.
     */
    public void addNumEndorse() {
        numEndorsements++;
    }

    /**
     * This method minuses one to the number of posts, including endorsements and replies the 
     * account has.
     */
    public void minusNumPost() {
        numPosts--;
    }

    /**
     * This method minuses one to the number of endorsements the account has.
     */
    public void minusNumEndorse() {
        numEndorsements--;
    }

    /**
     * This method returns the handle of the user.
     * 
     * @return The handle of the user.
     */
    public String getHandle() {
        return handle;
    }

    /**
     * This method returns the description connected to the account.
     * 
     * @return The description of the account.
     */
    public String getDescription() {
        return description;
    }
    
    /**
     * This method returns the id of the user.
     * 
     * @return The Id of the account is what gets returned.
     */
    public int getId() {
        return id;
    }

    /**
     * This method returns the number of posts, including endorsements and replies an account has.
     * 
     * @return the number of posts, including endorsements and replies an account has.
     */
    public int getNumPosts() {
        return numPosts;
    }

    /**
     * This method returns the number of endorsements an account has.
     * 
     * @return the number of endorsements an account has.
     */
    public int getNumEndorsements() {
        return numEndorsements;
    }

    /**
     * This method allows Accounts to be printed out as text that includes
     * all useful information.
     * 
     * @return The string of text is returned. 
     */
    public String toString() {
        String descToShow = "";
        if (description != null) {
            descToShow += description;
        }
        return "ID: " + id + "\nHandle: " + handle + "\nDescription: " + 
            descToShow+"\nPost count: "+numPosts+"\nEndorse count: "+numEndorsements;
    }
}
\end{minted}
\newpage
\subsection{Post}
\begin{minted}
[
frame=lines,
framesep=2mm,
baselinestretch=1.2,
fontsize=\footnotesize,
linenos,
obeytabs=true,
tabsize=2
]
{java}
package socialmedia;

import java.io.Serializable;

/**
 * This class consists of the methods and attributes of a Post and therefore is
 * responsible for all the details regarding a Post and due to how a Comment and 
 * Endorsement inherit from Post it is respinsible for some methods ad details of
 * a a Comment and Endorsement.
 * <p>
 * A Post had a unique sequencial ID, a message up to 100 characters long, an 
 * author (Account) it is associated with, a list of all comments it recieves 
 * and a list of all endorsements it recieves.
 * <p>
 * The class itself has two static values which are the serialVersionUID which is 
 * used when serializing the platform and numPosts which tracks the number of posts
 * which have been created.
 * 
 * @author 700008432
 * @author 690033172
 * @version 1.0
 */
public class Post implements Serializable {

    /**
     * The SerialVersionUID which represents the class version.
     */
    private static final long serialVersionUID = -6471649759148628316L;

    /**
     * This is the unique sequencial identifier of the post.
     */
    private int id;

    /**
     * This is the message with up to 100 characters.
     */
    private String message;

    /**
     * This is the Account associated with the post.
     */
    private Account author;

    /**
     * This is an array of all the comments the post receieves.
     */
    private Comment[] allComments = {};

    /**
     * This is an array of all the endorsements the post receieves.
     */
    private Endorsement[] allEndorsements = {};


    private boolean isActionable;

    /**
     * This is a staic integer value which records the number of posts in the system starting 
     * at 0, it is used to get the next sequencial ID for when a post is created.
     */
    private static int numPosts = 0;

    /**
     * This is the constructor method for a Post, it takes in the parameters author which refers
     * to the account associated with the post and message which must be up to 100 characters long
     * for a new post or comment.
     *  
     * @param author    The account associated with the post.
     * @param message   The message contained in the post.
     * 
     * @throws InvalidPostException             Thrown if the message is longer than 100 characters.
     * @throws HandleNotRecognisedException     Thrown if an Account with given handle does not exist.
     */
    public Post(Account author, String message)
                throws InvalidPostException,
                HandleNotRecognisedException {
        if (message.length() > 100) {
            throw new InvalidPostException("Post has a length of more than 100 characters.");
        }
        if (message.equals("")) {
            throw new InvalidPostException("Post cannot be empty.");
        }
        if (author == null) {
            throw new HandleNotRecognisedException("Account with given handle does not exist.");
        }
        this.author = author;
        this.message = message;
        this.id = numPosts++;
        this.isActionable = true;
    }

    /**
     * This is the constructor method for a Post, it takes in the parameters author which refers
     * to the account associated with the post and linkedPost which is the post the new post is associated
     * with when creating an endorsement post.
     * 
     * @param author        The account associated with the post.
     * @param linkedPost    The post associated with this endorsement post.
     * 
     * @throws HandleNotRecognisedException     Thrown if an Account with given handle does not exist.
     */
    public Post(Account author, Post linkedPost) 
                throws HandleNotRecognisedException {
        if (author == null) {
            throw new HandleNotRecognisedException("Account with given handle does not exist.");
        }
        this.author = author;
        String linkedAuthorName = linkedPost.getAuthor().getHandle();
        String linkedPostMessage = linkedPost.getMessage();
        if ((linkedPostMessage.length() + linkedAuthorName.length() + 5) > 100) {
            linkedPostMessage = linkedPostMessage.substring(0, 95-linkedAuthorName.length());
        }
        this.message = "EP@"+linkedAuthorName+": "+linkedPostMessage;
        this.id = numPosts++;
        this.isActionable = false;
    }

    /**
     * This methods adds a new comment to the post's list of comments it has recieved.
     * 
     * @param newComment    The new comment to add to its list.
     */
    public void addComment(Comment newComment) {
        Comment[] newList = new Comment[allComments.length + 1];
        for (int i = 0; i < allComments.length; i++) {
            newList[i] = allComments[i];
        }
        newList[allComments.length] = newComment;
        allComments = newList;
    }

    /**
     * This method adds a new endorsement to the post's list of endorsements it has recieved.
     * 
     * @param newEndorsement    The new endorsement to add to its list.
     */
    public void addEndorsement(Endorsement newEndorsement) {
        Endorsement[] newList = new Endorsement[allEndorsements.length + 1];
        for (int i = 0; i < allEndorsements.length; i++) {
            newList[i] = allEndorsements[i];
        }
        newList[allEndorsements.length] = newEndorsement;
        allEndorsements = newList;
    }

    /**
     * This method emptys a post. This includes removing the author a post refers to, replacing
     * the messsage with the string "The original content was removed from the system and is no longer
     * available." and making the post non-actioanble.
     */
    public void empty() {
        this.message = "The original content was removed from the system and is no longer available.";
        this.author = null;
        this.isActionable = false;
    }

    /**
     * This method resets the number of posts to 0.
     */
    public static void resetNumPosts() {
        numPosts = 0;
    }

    /**
     * This method returns the id of the Post.
     * 
     * @return The id of the Post.
     */
    public int getID() {
        return id;
    }

    /**
     * This method returns the message contained in the Post.
     * 
     * @return The message contained in the Post.
     */
    public String getMessage() {
        return message;
    }

    /**
     * This method returns the author of the Post.
     * 
     * @return The author associated with the Post.
     */
    public Account getAuthor() {
        return author;
    }

    /**
     * This method returns all the endorsement posts of the Post.
     * 
     * @return The array of all endorsements.
     */
    public Endorsement[] getAllEndorsements() {
        return allEndorsements;
    }

    /**
     * This method returns all the comments associated with the Post
     * 
     * @return The array of all comments.
     */
    public Comment[] getAllComments() {
        return allComments;
    }

    /**
     * This method returns if the post can be acted upon.
     * 
     * @return isActionable If the post can be acted upon.
     */
    public boolean getIsActionable() {
        return isActionable;
    }

    /**
     * This method returns the post in string format as defined in the MiniSocialMediaPlatform
     * interface.
     * 
     * @return The post in string format.
     */
    @Override
    public String toString() {
        String name = "";
        if (!(author == null)){
            name += author.getHandle();
        }
        return "ID: "+id+"\nAccount: "+name+"\nNo. endorsements: "+allEndorsements.length+
            " | No. comments: "+allComments.length+"\n"+message;
    }
}
\end{minted}
\newpage
\subsection{Comment}
\begin{minted}
[
frame=lines,
framesep=2mm,
baselinestretch=1.2,
fontsize=\footnotesize,
linenos,
obeytabs=true,
tabsize=2
]
{java}
package socialmedia;

/**
 * This class consists of the specific methods and attributes of a Comment, some of the other methods and
 * attributes of a Comment are inheritted from a Post. Therefore this class is responsible for the Comment
 * specific methods and attributes.
 * <p>
 * As a Comment inherits from a Post and thefore has access to all of a Post's attributes 
 * and methods. In addition to these, a Comment has a linkedPost attribute which refers
 * to the Post which the comment has been placed on.
 * <p>
 * The class itself has a static value for its SerialVersionUID used when 
 * serializing the platform.
 * 
 * @author 700008432
 * @author 690033172
 * @version 1.0
 */
public class Comment extends Post{

    /**
     * The SerialVersionUID which represents the class version.
     */
    private static final long serialVersionUID = -7709952939872791336L;
    
    /**
     * This is the post the comment is associated with.
     */
    private Post linkedPost;

    /**
     * This is the constructor method for a Comment, it takes in the parameters author which refers
     * to the account associated with the post, message which must be up to 100 characters long and
     * a Post which the comment is associated with.
     * 
     * @param author        The account associated with the comment.
     * @param message       The message contained in the comment.
     * @param linkedPost    The Post which the comment has ben added on.
     * 
     * @throws InvalidPostException             Thrown when the message is longer than 100 characters.
     * @throws HandleNotRecognisedException     Thrown when the given account ahdnle does not exist in
     *                                          the system.
     */
    public Comment(Account author, String message, Post linkedPost) 
                    throws InvalidPostException,
                    HandleNotRecognisedException {
        super(author, message);
        this.linkedPost = linkedPost;
    }

    /**
     * This method is used to change the post which the comment is associated with.
     * 
     * @param newLinkedPost     The new post the comment is associated with.
     */
    public void changeLinkedPost(Post newLinkedPost) {
        this.linkedPost = newLinkedPost;
    }

    /**
     * This method returns the Post the comment is associated with.
     * 
     * @return The post the comment is associated with.
     */
    public Post getLinkedPost() {
        return linkedPost;
    }
}
\end{minted}
\newpage
\subsection{Endorsement}
\begin{minted}
[
frame=lines,
framesep=2mm,
baselinestretch=1.2,
fontsize=\footnotesize,
linenos,
obeytabs=true,
tabsize=2
]
{java}
package socialmedia;

/**
 * This class consists of the specific methods and attributes of a Endorsement, some of the
 * other methods and attributes of a Endorsement are inheritted from a Post. Therefore this
 * class is responsible for only the methods and attributes unqiue for an Endorsement.
 * <p>
 * As an Endorsement inherits from a Post and thefore has access to all of a Post's attributes 
 * and methods. In addition to these, an Endorsement has a linkedPost attribute which refers
 * to the Post which the endorsement has been placed on and replicates the message of the post
 * it is associated with.
 * <p>
 * The class itself also has a static value for its SerialVersionUID used when 
 * serializing the platform.
 * 
 * @author 700008432
 * @author 690033172
 * @version 1.0
 */
public class Endorsement extends Post {

    /**
     * The SerialVersionUID which represents the class version.
     */
    private static final long serialVersionUID = 8178673147762741699L;
    
    /**
     * This is the post the comment is associated with.
     */
    private Post linkedPost;

    /**
     * This is the constructor method for an Endorsement, it takes in the parameters author which refers
     * to the account associated with the post and a Post which the endorsement is associated with.
     * The message of the post it is associated wth is formatted as:
     * 
     * <p>
	 * <code>"EP@" + [endorsed account handle] + ": " + [endorsed message]</code>
	 * <p>
     * 
     * @param author        The account associated with the endorsement.
     * @param linkedPost    The Post which the endorsement has ben added on.
     * 
     * @throws HandleNotRecognisedException     Thrown if an Account with given handle does not exist.
     */
    public Endorsement(Account author, Post linkedPost)
                        throws HandleNotRecognisedException {
        super(author, linkedPost);
        this.linkedPost = linkedPost;
    }

    /**
     * This method returns the Post the endorsement is associated with.
     * 
     * @return The post the endorsement is associated with.
     */
    public Post getLinkedPost() {
        return linkedPost;
    }
}
\end{minted}
\newpage
\subsection{SocialMedia}
\begin{minted}
[
frame=lines,
framesep=2mm,
baselinestretch=1.2,
fontsize=\footnotesize,
linenos,
obeytabs=true,
tabsize=2
]
{java}
package socialmedia;

import java.io.FileInputStream;
import java.io.FileOutputStream;
import java.io.IOException;
import java.io.ObjectInputStream;
import java.io.ObjectOutputStream;

/**
 * This class is am implimetation of the SocialMediaPlatform interface which extends the 
 * MiniSocialMediaPlatform interface. Therefore it is resposible for most of the functionality 
 * of the platform. 
 * <p>
 * A SocialMedia object has a list of all accounts and a list of all posts which is used to hold
 * all data on the Social Media Platform. The accounts, posts, endorsments and comments in these
 * lists are then created, removed an edisted using the methods in the class.
 * <p>
 * The class itself also has a static value for its SerialVersionUID used when 
 * serializing the platform.
 * 
 * @author 700008432
 * @author 690033172
 * @version 1.0
 */
public class SocialMedia implements SocialMediaPlatform {

	/**
	 * The SerialVersionUID which represents the class version.
	 */
	private static final long serialVersionUID = -5306444784233657143L;

	/**
	 * This is an array of all accounts on the platform.
	 */
	private Account[] allAccounts = {};

	/**
	 * This is an array of all posts on the platform.
	 */
	private Post[] allPosts = {};

	/**
	 * This method is used to add a instance of an account to the array of all accounts on
	 * the platform.
	 * 
	 * @param newAccount	The new account to be added to the array.
	 */
	public void addAccountToAllAccounts(Account newAccount) {
		Account[] newList = new Account[allAccounts.length + 1];
		for (int i = 0; i < allAccounts.length; i++) {
			newList[i] = allAccounts[i];
		}
		newList[allAccounts.length] = newAccount;
		allAccounts = newList;
	}

	/**
	 * This method is used to delete an account from all the accounts in the system.
	 * 
	 * @param deleteAccount 	The account to be deleted.
	 */
	public void deleteAccountFromAllAccounts(Account deleteAccount) {
		Account[] newList = new Account[allAccounts.length - 1];
		for (int i = 0, j = 0; i < allAccounts.length; i++) {
			if (!allAccounts[i].equals(deleteAccount)) {
				newList[j] = allAccounts[i];
				j++;
			}
		}
		allAccounts = newList;
	}

	/**
	 * This method is used to add a instance of a post to the array of all posts on
	 * the platform.
	 * 
	 * @param newPost	The new post to be added to the array.
	 */
	public void addPostToAllPosts(Post newPost) {
		Post[] newList = new Post[allPosts.length + 1];
		for (int i = 0; i < allPosts.length; i++) {
			newList[i] = allPosts[i];
		}
		newList[allPosts.length] = newPost;
		allPosts = newList;
	}

	/**
	 * This method is used to delete a post from all the posts in the system.
	 * 
	 * @param deletePost 	The post to be deleted.
	 */
	public void deletePostFromAllPosts(Post deletePost) {
		Post[] newList = new Post[allPosts.length - 1];
		for (int i = 0, j = 0; i < allPosts.length; i++) {
			if (!allPosts[i].equals(deletePost)) {
				newList[j] = allPosts[i];
				j++;
			}
		}
		allPosts = newList;
	}

	/**
	 * This method gets a Post from all posts with the given id.
	 * 
	 * @param id 	The id of the wanted post.
	 * 
	 * @return wantedPost The post with the given id.
	 */
	public Post getPost(int id) {
		Post wantedPost = null;
		for (Post post : allPosts) {
			if (post.getID() == id) {
				wantedPost = post;
				break;
			}
		}
		return wantedPost;
	}
	
	/**
	 * This method gets an Account from all accounts with the given handle.
	 * 
	 * @param handle	The handle of the desired account.
	 * 
	 * @return	This returns the account with the matching handle.
	 */
	public Account getAccount(String handle) {
		Account wantedAccount = null;
		for (Account account : allAccounts) {
			if (account.getHandle().equals(handle)) {
				wantedAccount = account;
				break;
			}
		}
		return wantedAccount;
	}

	/**
	 * This method gets an Account from all accounts with the given id.
	 * 
	 * @param id	The id of the desired account.
	 * 
	 * @return	This returns the account with the matching id.
	 */
	public Account getAccount(int id) {
		Account wantedAccount = null;
		for (Account account : allAccounts) {
			if (account.getId() == id) {
				wantedAccount = account;
				break;
			}
		}
		return wantedAccount;
	}

    /**
     * This method allows the user to create an account with just their handle
	 * and it returns their id number afterwards. It does the checks on the handle
	 * to ensure that there are no other users with the same handle, the handle is 
	 * under 30 characters, is not empty and finally that it has no white space.
	 * 
	 * @param handle	This is the handle the user wants to use
	 * 
	 * @throws IllegalHandleException	Thrown if the handle is already in use by another user.
	 * @throws InvalidHandleException	Thrown if the handle is over 30 chars, is empty or has white space.
	 * 
	 * @return This returns the id of the new account. 
	 */
    @Override
	public int createAccount(String handle)
								throws IllegalHandleException,
								InvalidHandleException {
		Account newAccount = new Account(handle);
		addAccountToAllAccounts(newAccount);
		return newAccount.getId();
	}

    /**
     * This method allows the user to create an account with the handle and description.
	 * It does checks to ensure the handle is valid and not already in use within the
	 * system.
	 * 
	 * @param handle		The handle of the new user.
	 * @param description	The description to be displayed on the profile.
	 * 
	 * @throws IllegalHandleException	Thrown if the handle is already in use by another user.
	 * @throws InvalidHandleException	Thrown if the handle is over 30 chars, is empty or has white space.
	 * 
	 * @return This returns the id of the new account. 
     */
	@Override
	public int createAccount(String handle, String description)
								throws IllegalHandleException,
								InvalidHandleException {
		Account newAccount = new Account(handle, description);
		addAccountToAllAccounts(newAccount);
		return newAccount.getId();
	}

    /**
     * This method allows the user to delete an account using the account id.
	 * 
	 * @param id 	The id related to account to be deleted.
	 * 
	 * @throws AccountIDNotRecognisedException	Thrown when the account id used is not in the system.
     */
	@Override
	public void removeAccount(int id)
								throws AccountIDNotRecognisedException {
		Account deleteAccount = getAccount(id);

		if (deleteAccount == null) {
			throw new AccountIDNotRecognisedException("Attemting to delete Account with a given 
			    Account ID which does not exist in system.");
		}
		
		for (Post post : allPosts) {
			if (post.getAuthor() == deleteAccount) {
				deletePostFromAllPosts(post); //deletes each post before removing the account.
			}
		}
		deleteAccountFromAllAccounts(deleteAccount);

	}

    /**
     * This method allows the user to remove an account using the account handle.
	 * 
	 * @param handle	The handle of the account to be removed.
	 * 
	 * @throws HandleNotRecognisedException	Thrown when the provided handle is not in the system.
     */
	@Override
	public void removeAccount(String handle)
								throws HandleNotRecognisedException {
		Account deleteAccount = getAccount(handle);

		if (deleteAccount == null) {
			throw new HandleNotRecognisedException("The account handle used when trying to delete account 
			    does not exist in system.");
		}

		for (Post post : allPosts) {
			if (post.getAuthor().equals(deleteAccount)) {
				for (Endorsement endorsement : post.getAllEndorsements()) {
					deletePostFromAllPosts(endorsement);
				}
				post.empty();
			}
		}
		deleteAccountFromAllAccounts(deleteAccount);
	}

    /**
     * This method allows a user to change their account handle.
	 * 
	 * @param oldHandle		The handle of the user already in the system.
	 * @param newHandle		The handle the user would like to change to.
	 * 
	 * @throws HandleNotRecognisedException 	Thrown when the old handle is not recognised in the system.
	 * @throws IllegalHandleException 			Thrown when the new handle is already in use by another account.
	 * @throws InvalidHandleException 			Thrown when the new handle is over 30 chars, empty, 
	 *                                          or contains white space.
     */
	@Override
	public void changeAccountHandle(String oldHandle, String newHandle)
									throws HandleNotRecognisedException,
									IllegalHandleException,
									InvalidHandleException {

		Account changeHandle = getAccount(oldHandle);

		for (Account account : allAccounts) {
			if (account.getHandle() == newHandle) {
				throw new IllegalHandleException("Handle is already in use in the system. ");
			}
		}

		if (changeHandle == null) {
			throw new HandleNotRecognisedException("The old account handle used does not exist in the system.");
		}
		
		changeHandle.changeHandle(newHandle);
	}

    /**
     * This method allows the user to update their account description.
	 * 
	 * @param handle		The handle of the account.
	 * @param description	The new description to be displayed.
	 * 
	 * @throws HandleNotRecognisedException Thrown when the handle is not recognised as belonging 
	 * to an account in the system.
     */
	@Override
	public void updateAccountDescription(String handle, String description) 
										throws HandleNotRecognisedException {

		Account accountDescrip = getAccount(handle);
		if (accountDescrip == null) {
			throw new HandleNotRecognisedException("The account handle used was not recognised by the system.");
		}
		accountDescrip.updateDescription(description);
	}

    /**
	 * This method allows an account selected with the handle to be displayed as a string, 
	 * giving all the important info.
	 * 
	 * @param handle	The handle of the account to be displayed
	 * 
	 * @throws HandleNotRecognisedException Thrown when the handle provided is not in the system.
	 * 
	 * @return Returns the account information as a string.
	 */
	@Override
	public String showAccount(String handle)
								throws HandleNotRecognisedException {

		Account wantedAccount = getAccount(handle);

		if (wantedAccount == null) {
			throw new HandleNotRecognisedException("The handle used is not recognised in the system");
		}

		String accountDisplayed = wantedAccount.toString();
		return accountDisplayed;
	}

    /**
     * This method impliments the SocialMediaPlatform method createPost by creating a post
	 * for the account identified by the given handle with the following message.
	 * 
	 * @param 	handle  The handle of the account the post is associated with.
	 * @param 	message The message associated with the post.
	 * 
	 * @throws HandleNotRecognisedException Thrown If the handle does not match to any
	 *                                      account in the system.
	 * @throws InvalidPostException         Thrown if the message is empty or has more than
	 *                                      100 characters.
	 * 
	 * @return The sequential ID of the created post.
     */
	@Override
	public int createPost(String handle, String message)
							throws HandleNotRecognisedException,
							InvalidPostException {
		Account author = getAccount(handle);
		Post newPost = new Post(author, message);

		addPostToAllPosts(newPost);

		author.addNumPost();
		
		return newPost.getID();
	}

    /**
     * This method impliments the SocialMediaPlatform method endorsePost by creating an
	 * endorsement post of an existing post, similar to a retweet on Twitter. An 
	 * endorsement post is a special post. It contains a reference to the endorsed post
	 * and its message is formatted as:
	 * <p>
	 * <code>"EP@" + [endorsed account handle] + ": " + [endorsed message]</code>
	 * <p>
	 * 
	 * @param handle 	The handle of the account associated with the endorsement.
	 * @param id		The id of the post associated with the endorsement.
	 * 
	 * @throws HandleNotRecognisedException		Thrown when the handle is not associated with a account
	 * 											on the platform.
	 * @throws PostIDNotRecognisedException		Thrown when the post id is not associated with a post on
	 * 											the platform.
	 * @throws NotActionablePostException		Thrown when attempting to act upon a non actionabe post.
	 * 
	 * @return The sequencial ID of the created endorsement.
     */
	@Override
	public int endorsePost(String handle, int id)
							throws HandleNotRecognisedException,
							PostIDNotRecognisedException,
							NotActionablePostException {
		Account author = getAccount(handle);
		Post linkedPost = getPost(id);

		if (linkedPost == null) {
			throw new PostIDNotRecognisedException("Post ID is not associated with Post in platform.");
		}

		if (!linkedPost.getIsActionable()) {
			throw new NotActionablePostException("Attempting to endorse a non-actionable post.");
		}

		Endorsement newEndorsement = new Endorsement(author, linkedPost);
		linkedPost.addEndorsement(newEndorsement);
		addPostToAllPosts(newEndorsement);
		linkedPost.getAuthor().addNumEndorse();
		linkedPost.getAuthor().addNumPost();
		author.addNumPost();
		return newEndorsement.getID();
	}

    /**
	 * This method impliments the SocialMediaPlatform method commentPost by creating a comment post 
	 * referring to an existing post, similarly to a reply on Twitter. A comment post is a special
	 * post. It contains a reference to the post being commented upon.
	 * 
	 * @param handle 	The handle of the account associated with the endorsement.
	 * @param id		The id of the post associated with the endorsement.
	 * @param message 	The message conatined in the comment.
	 * 
	 * @throws HandleNotRecognisedException		Thrown when the handle is not associated with a account
	 * 											on the platform.
	 * @throws PostIDNotRecognisedException		Thrown when the post id is not associated with a post on
	 * 											the platform.
	 * @throws NotActionablePostException		Thrown when attempting to act upon a non actionabe post.
	 * @throws InvalidPostException 			Thrown if the message is empty or has more than
	 *                                      	100 characters.	
	 * 
	 * @return The sequencial ID of the created comment.
	 */
	@Override
	public int commentPost(String handle, int id, String message)
							throws HandleNotRecognisedException,
							PostIDNotRecognisedException,
							NotActionablePostException,
							InvalidPostException {
		Account author = getAccount(handle);
		Post linkedPost = getPost(id);

		if (linkedPost == null) {
			throw new PostIDNotRecognisedException("Post ID is not associated with Post in platform.");
		}

		if (!linkedPost.getIsActionable()) {
			throw new NotActionablePostException("Attempting to comment on a non-actionable post.");
		}

		Comment newComment = new Comment(author, message, linkedPost);
		linkedPost.addComment(newComment);
		addPostToAllPosts(newComment);
		author.addNumPost();
		linkedPost.getAuthor().addNumPost();
		return newComment.getID();
	}

    /**
     * This method impliments the deletePost method of the MiniSocialMediaPlatform interface 
	 * by removing the post from the platform. When a post is removed, all
	 * its endorsements should be removed as well. All replies to this post should
	 * be updated by replacing the reference to this post by a generic empty post.
	 * <p>
	 * The generic empty post message should be "The original content was removed
	 * from the system and is no longer available.". This empty post is just a
	 * replacement placeholder for the post which a reply refers to. Empty posts
	 * should not be linked to any account and cannot be acted upon, i.e., it cannot
	 * be available for endorsements or replies.
	 * <p>
	 * 
	 * @param id	The id of the post to be displayed.
	 * 
	 * @throws PostIDNotRecognisedException		Thrown when the post id is not associated with a post on
	 * 											the platform.
     */
	@Override
	public void deletePost(int id) 
							throws PostIDNotRecognisedException {
		Post deletePost = getPost(id);

		if (deletePost == null) {
			throw new PostIDNotRecognisedException("Post ID of post to be deleted not in the system.");
		}

		for (Endorsement endorsement : deletePost.getAllEndorsements()) {
			deletePostFromAllPosts(endorsement);
			endorsement.getAuthor().minusNumPost();
			endorsement.getLinkedPost().getAuthor().minusNumEndorse();
			endorsement.getLinkedPost().getAuthor().minusNumPost();
		}

		deletePost.getAuthor().minusNumPost();

		if (deletePost.getClass().getName().startsWith("socialmedia.Endorsement")) {
			((Endorsement) deletePost).getLinkedPost().getAuthor().minusNumEndorse();
			((Endorsement) deletePost).getLinkedPost().getAuthor().minusNumPost();
		}
		else if (deletePost.getClass().getName().startsWith("socialmedia.Comment")) {
			((Comment) deletePost).getLinkedPost().getAuthor().minusNumPost();
		}

		deletePost.empty();
	}

    /**
     * This method impliments the showIndividualPost method by finding the Post which coresponds with the
	 * given post ID and the getting the descirption from the post's toString method. The format is
	 * as follows:
	 * 
	 * <pre>
	 * ID: [post ID]
	 * Account: [account handle]
	 * No. endorsements: [number of endorsements received by the post] | 
	 * No. comments: [number of comments received by the post]
	 * [post message]
	 * </pre>
	 * 
	 * @param id		The id of the post to be displayed.
	 * 
	 * @throws PostIDNotRecognisedException		Thrown when the post id is not associated with a post on
	 * 											the platform.
	 * 
	 * @return The description of the Post.
     */
	@Override
	public String showIndividualPost(int id)
										throws PostIDNotRecognisedException {
		Post wantedPost = getPost(id);

		if (wantedPost == null) {
			throw new PostIDNotRecognisedException("Post with given id does not exist in system.");
		}

		String description = wantedPost.toString();

		return description;
	}

    /**
     * This method impliments the showPostChildrenDetails of the MiniSocialMediaPlatform 
	 * interface by building a StringBuilder showing the details of the current post and
	 * all its children posts. See interface for further details.
	 * 
	 * @param id 	The id of the post to be shown.
	 * 
	 * @throws PostIDNotRecognisedException 	Thrown if the ID does not match to any post in
	 *                                      	the system.
	 * @throws NotActionablePostException   	Thrown if the ID refers to an endorsement post.
	 *                                      	Endorsement posts do not have children
	 *                                      	since they are not endorsable nor
	 *                                      	commented.
	 * 
	 * @return A formatted StringBuilder containing the details of the post and its
	 *         children.
     */
	@Override
	public StringBuilder showPostChildrenDetails(int id)
												throws PostIDNotRecognisedException,
												NotActionablePostException {
		StringBuilder postDetails = new StringBuilder();

		Post wantedPost = getPost(id);

		if (wantedPost == null) {
			throw new PostIDNotRecognisedException("Post with given id does not exist in system.");
		}

		if (!wantedPost.getIsActionable()) {
			throw new NotActionablePostException("Attempting to view non-actionable post.");
		}

		postDetails.append(showIndividualPost(id));
		postDetails.append("\n | ");
		addPostChildrenDetails(postDetails, wantedPost.getAllComments(), 0);

		return postDetails;
	}

	/**
	 * This is method which uses recurssion to display the posts and all the sub-posts. 
	 * 
	 * @param postDetails	The StringBuilder to add the posts details to.
	 * @param comments		The comments to loop over adding them to the BuildString
	 * 						and recursively calling the method again for any subcomments.
	 * @param numTabs		The number of tabs to add for this set of comments.
	 * 
	 * @throws PostIDNotRecognisedException 	Thrown if the ID does not match to any post in
	 *                                      	the system.
	 */
	public void addPostChildrenDetails(StringBuilder postDetails, Comment[] comments, int numTabs) 
										throws PostIDNotRecognisedException {
		for (Comment comment : comments) {
			String tabs = new String(new char[numTabs]).replace("\0", "     ");
			postDetails.append("\n"+tabs+" | > "+showIndividualPost(comment.getID()).replace("\n", "\n     "+tabs));
			if (comment.getAllComments().length != 0) {
				tabs = new String(new char[numTabs + 1]).replace("\0", "     ");
				postDetails.append("\n"+tabs+" | ");
				addPostChildrenDetails(postDetails, comment.getAllComments(), numTabs+1);
			}
			else if (numTabs == 0){
				postDetails.append("\n");
			}
			
		}
	}

    /**
     * This method is an implimentation of the getNumberAccounts method in the 
	 * SocialMediaPlatform interface by returning the current total number of accounts
	 * present in the platform. Note, this is NOT the total number of accounts ever 
	 * created since the current total should discount deletions.
	 * 
	 * @return The total number of accounts in the platform.
     */
	@Override
	public int getNumberOfAccounts() {
		int numAccounts = allAccounts.length;
		return numAccounts;
	}

    /**
     * This method is an implimentation of the getTotalOriginalPosts method in the
	 * SocialMediaPlatform interface by returning the current total number of original
	 * posts (i.e., disregarding endorsements and comments) present in the platform. 
	 * Note, this is NOT the total number of posts ever created since the current total
	 * should discount deletions.
	 * 
	 * @return The total number of original posts in the platform.
     */
	@Override
	public int getTotalOriginalPosts() {
		int numoriginalPosts = 0;
		for (Post post : allPosts) {
			if (post.getClass().getName().startsWith("socialmedia.Post")) {
				numoriginalPosts++;
			}
		}
		return numoriginalPosts;
	}

    /**
     * This method is an implimentation of the getTotalEndorsementPost method in the
	 * SocialMediaPlatform interface by returning the current total number of endorsement
	 * posts present in the platform. Note, this is NOT the total number of endorsements 
	 * ever created since the current total should discount deletions.
	 * 
	 * @return The total number of endorsement posts in the platform.
     */
	@Override
	public int getTotalEndorsmentPosts() {
		int numEndorsementPosts = 0;
		for (Post post : allPosts) {
			if (post.getClass().getName().startsWith("socialmedia.Endorsement")) {
				numEndorsementPosts++;
			}
		}
		return numEndorsementPosts;
	}
    
    /**
     * This method is an implimentation of the getTotalCommentPosts method in the 
	 * SocialMediaPlatform interface by returning the current total number of comments posts
	 * present in the platform. Note, this is NOT the total number of comments ever created 
	 * since the current total should discount deletions.
	 * 
	 * @return The total number of comments posts in the platform.
     */
	@Override
	public int getTotalCommentPosts() {
		int numCommentPosts = 0;
		for (Post post : allPosts) {
			if (post.getClass().getName().startsWith("socialmedia.Comment")) {
				numCommentPosts++;
			}
		}
		return numCommentPosts;
	}

    /**
     * This method implients the getMostEndorsedPost method in the MiniSocialMediaPlatform
	 * iterface by identifing and returning the post with the most number of endorsements,
	 * a.k.a. the most popular post.
	 * <p>
	 * If the method returns -1 then it means there are no Posts in the platform.
	 * 
	 * @return The ID of the most popular post.
     */
	@Override
	public int getMostEndorsedPost() {
		Post mostEndorsed = null;
		int maxNumEndorsements = 0;
		int postNumEndorsements;
		int mostEndorsedId = -1;
		for (Post post : allPosts) {
			postNumEndorsements = post.getAllEndorsements().length;
			if (postNumEndorsements >= maxNumEndorsements) {
				mostEndorsed = post;
				maxNumEndorsements = postNumEndorsements;
			}
		}
		if (mostEndorsed != null) {
			mostEndorsedId = mostEndorsed.getID();
		}

		return mostEndorsedId;
	}

	/**
	 * This method allows the user to retrieve the account with most endorsements. It loops over every
	 * account within the system and checks over all their posts to see how many endorsements each user
	 * has.
	 * <p>
	 * If the method returns -1 then it means there are no Accounts in the platform.
	 * 
	 * @return It returns the Id of the account with the most endorsements on the platform.
	 */
	@Override
	public int getMostEndorsedAccount() {
		Account mostEndorsedAccount = null;
		int mostEndorsedAccountId = -1;
		int maxNumEndorsements = 0;
		int numEndorse;

		for (Account account : allAccounts) {
			numEndorse = account.getNumEndorsements();
			if (numEndorse > maxNumEndorsements) {
				mostEndorsedAccount = account;
				maxNumEndorsements = numEndorse;
			}
		}
		if (mostEndorsedAccount != null) {
			mostEndorsedAccountId = mostEndorsedAccount.getId();
		}
		return mostEndorsedAccountId;
	}

    /**
     * This method impliments the erasePlatform method in the MiniSocialMediaPlatform 
	 * interface by emptying this SocialMediaPlatform of its contents and resets all
	 * internal counters in Account and Post.
     */
	@Override
	public void erasePlatform() {
		allAccounts = new Account[] {};
		allPosts = new Post[] {};
		Post.resetNumPosts();
		Account.resetNumAccounts();
	}

    /**
     * This method impliments the savePlatform method in the MiniSocialMediaPlatform 
	 * interace by saving this SocialMediaPlatform's contents into a serialised file, with
	 * the filename given in the argument.
	 *
	 * @param filename The path and name of the file to be saved to.
	 * 
	 * @throws IOException Thrown if there is a problem experienced when trying to save the
	 *                     store contents to the file.
     */
	@Override
	public void savePlatform(String filename) 
							throws IOException {
		FileOutputStream fileOut = new FileOutputStream(filename);
		ObjectOutputStream out = new ObjectOutputStream(fileOut);
		out.writeObject(this.allAccounts);
		out.writeObject(this.allPosts);
		out.close();
		fileOut.close();
	}

    /**
     * This method impliments the loadPlatform method in the MiniSocialMediaPlatform 
	 * interface by loading and replacing this SocialMediaPlatform's contents with the
	 * serialised contents stored in the file given in the argument.
	 *
	 * @param filename The location of the file to be loaded.
	 * 
	 * @throws IOException            Thrown if there is a problem experienced when trying
	 *                                to load the store contents from the file.
	 * @throws ClassNotFoundException Thrown if required class files cannot be found when
	 *                                loading.
     */
	@Override
	public void loadPlatform(String filename) 
							throws IOException,
							ClassNotFoundException {
		FileInputStream fileIn = new FileInputStream(filename);
		ObjectInputStream in = new ObjectInputStream(fileIn);
		allAccounts = (Account[]) in.readObject();
		allPosts = (Post[]) in.readObject();
		in.close();
		fileIn.close();
	}
}
\end{minted}
\newpage
\end{document}
